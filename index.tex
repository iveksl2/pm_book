% Options for packages loaded elsewhere
\PassOptionsToPackage{unicode}{hyperref}
\PassOptionsToPackage{hyphens}{url}
\PassOptionsToPackage{dvipsnames,svgnames,x11names}{xcolor}
%
\documentclass[
  letterpaper,
  DIV=11,
  numbers=noendperiod]{scrreprt}

\usepackage{amsmath,amssymb}
\usepackage{lmodern}
\usepackage{iftex}
\ifPDFTeX
  \usepackage[T1]{fontenc}
  \usepackage[utf8]{inputenc}
  \usepackage{textcomp} % provide euro and other symbols
\else % if luatex or xetex
  \usepackage{unicode-math}
  \defaultfontfeatures{Scale=MatchLowercase}
  \defaultfontfeatures[\rmfamily]{Ligatures=TeX,Scale=1}
\fi
% Use upquote if available, for straight quotes in verbatim environments
\IfFileExists{upquote.sty}{\usepackage{upquote}}{}
\IfFileExists{microtype.sty}{% use microtype if available
  \usepackage[]{microtype}
  \UseMicrotypeSet[protrusion]{basicmath} % disable protrusion for tt fonts
}{}
\makeatletter
\@ifundefined{KOMAClassName}{% if non-KOMA class
  \IfFileExists{parskip.sty}{%
    \usepackage{parskip}
  }{% else
    \setlength{\parindent}{0pt}
    \setlength{\parskip}{6pt plus 2pt minus 1pt}}
}{% if KOMA class
  \KOMAoptions{parskip=half}}
\makeatother
\usepackage{xcolor}
\setlength{\emergencystretch}{3em} % prevent overfull lines
\setcounter{secnumdepth}{5}
% Make \paragraph and \subparagraph free-standing
\ifx\paragraph\undefined\else
  \let\oldparagraph\paragraph
  \renewcommand{\paragraph}[1]{\oldparagraph{#1}\mbox{}}
\fi
\ifx\subparagraph\undefined\else
  \let\oldsubparagraph\subparagraph
  \renewcommand{\subparagraph}[1]{\oldsubparagraph{#1}\mbox{}}
\fi


\providecommand{\tightlist}{%
  \setlength{\itemsep}{0pt}\setlength{\parskip}{0pt}}\usepackage{longtable,booktabs,array}
\usepackage{calc} % for calculating minipage widths
% Correct order of tables after \paragraph or \subparagraph
\usepackage{etoolbox}
\makeatletter
\patchcmd\longtable{\par}{\if@noskipsec\mbox{}\fi\par}{}{}
\makeatother
% Allow footnotes in longtable head/foot
\IfFileExists{footnotehyper.sty}{\usepackage{footnotehyper}}{\usepackage{footnote}}
\makesavenoteenv{longtable}
\usepackage{graphicx}
\makeatletter
\def\maxwidth{\ifdim\Gin@nat@width>\linewidth\linewidth\else\Gin@nat@width\fi}
\def\maxheight{\ifdim\Gin@nat@height>\textheight\textheight\else\Gin@nat@height\fi}
\makeatother
% Scale images if necessary, so that they will not overflow the page
% margins by default, and it is still possible to overwrite the defaults
% using explicit options in \includegraphics[width, height, ...]{}
\setkeys{Gin}{width=\maxwidth,height=\maxheight,keepaspectratio}
% Set default figure placement to htbp
\makeatletter
\def\fps@figure{htbp}
\makeatother
\newlength{\cslhangindent}
\setlength{\cslhangindent}{1.5em}
\newlength{\csllabelwidth}
\setlength{\csllabelwidth}{3em}
\newlength{\cslentryspacingunit} % times entry-spacing
\setlength{\cslentryspacingunit}{\parskip}
\newenvironment{CSLReferences}[2] % #1 hanging-ident, #2 entry spacing
 {% don't indent paragraphs
  \setlength{\parindent}{0pt}
  % turn on hanging indent if param 1 is 1
  \ifodd #1
  \let\oldpar\par
  \def\par{\hangindent=\cslhangindent\oldpar}
  \fi
  % set entry spacing
  \setlength{\parskip}{#2\cslentryspacingunit}
 }%
 {}
\usepackage{calc}
\newcommand{\CSLBlock}[1]{#1\hfill\break}
\newcommand{\CSLLeftMargin}[1]{\parbox[t]{\csllabelwidth}{#1}}
\newcommand{\CSLRightInline}[1]{\parbox[t]{\linewidth - \csllabelwidth}{#1}\break}
\newcommand{\CSLIndent}[1]{\hspace{\cslhangindent}#1}

\KOMAoption{captions}{tableheading}
\makeatletter
\makeatother
\makeatletter
\@ifpackageloaded{bookmark}{}{\usepackage{bookmark}}
\makeatother
\makeatletter
\@ifpackageloaded{caption}{}{\usepackage{caption}}
\AtBeginDocument{%
\ifdefined\contentsname
  \renewcommand*\contentsname{Table of contents}
\else
  \newcommand\contentsname{Table of contents}
\fi
\ifdefined\listfigurename
  \renewcommand*\listfigurename{List of Figures}
\else
  \newcommand\listfigurename{List of Figures}
\fi
\ifdefined\listtablename
  \renewcommand*\listtablename{List of Tables}
\else
  \newcommand\listtablename{List of Tables}
\fi
\ifdefined\figurename
  \renewcommand*\figurename{Figure}
\else
  \newcommand\figurename{Figure}
\fi
\ifdefined\tablename
  \renewcommand*\tablename{Table}
\else
  \newcommand\tablename{Table}
\fi
}
\@ifpackageloaded{float}{}{\usepackage{float}}
\floatstyle{ruled}
\@ifundefined{c@chapter}{\newfloat{codelisting}{h}{lop}}{\newfloat{codelisting}{h}{lop}[chapter]}
\floatname{codelisting}{Listing}
\newcommand*\listoflistings{\listof{codelisting}{List of Listings}}
\makeatother
\makeatletter
\@ifpackageloaded{caption}{}{\usepackage{caption}}
\@ifpackageloaded{subcaption}{}{\usepackage{subcaption}}
\makeatother
\makeatletter
\@ifpackageloaded{tcolorbox}{}{\usepackage[many]{tcolorbox}}
\makeatother
\makeatletter
\@ifundefined{shadecolor}{\definecolor{shadecolor}{rgb}{.97, .97, .97}}
\makeatother
\makeatletter
\makeatother
\ifLuaTeX
  \usepackage{selnolig}  % disable illegal ligatures
\fi
\IfFileExists{bookmark.sty}{\usepackage{bookmark}}{\usepackage{hyperref}}
\IfFileExists{xurl.sty}{\usepackage{xurl}}{} % add URL line breaks if available
\urlstyle{same} % disable monospaced font for URLs
\hypersetup{
  pdftitle={pm\_book},
  pdfauthor={Igor Veksler},
  colorlinks=true,
  linkcolor={blue},
  filecolor={Maroon},
  citecolor={Blue},
  urlcolor={Blue},
  pdfcreator={LaTeX via pandoc}}

\title{pm\_book}
\author{Igor Veksler}
\date{12/25/22}

\begin{document}
\maketitle
\ifdefined\Shaded\renewenvironment{Shaded}{\begin{tcolorbox}[sharp corners, borderline west={3pt}{0pt}{shadecolor}, boxrule=0pt, interior hidden, enhanced, breakable, frame hidden]}{\end{tcolorbox}}\fi

\renewcommand*\contentsname{Table of contents}
{
\hypersetup{linkcolor=}
\setcounter{tocdepth}{2}
\tableofcontents
}
\bookmarksetup{startatroot}

\hypertarget{preface}{%
\chapter*{Preface}\label{preface}}
\addcontentsline{toc}{chapter}{Preface}

\markboth{Preface}{Preface}

This is a Quarto book.

To learn more about Quarto books visit
\url{https://quarto.org/docs/books}.

\bookmarksetup{startatroot}

\hypertarget{introduction}{%
\chapter{Introduction}\label{introduction}}

This project started as a
\href{https://github.com/iveksl2/product_management_links}{collection of
product management links} I collected as product management
best-practices. The links grew and this is a longer form book to provide
deeper explanations on the concetps. I am writing the book partly for
myself (in being to surface gaps in my knowledge) but I hope others
embarking on a career in product management will find some of the ideas
useful.

I am of the belief that the best mindset one can adopt as a product
manager is one of an entreprenuer. This book will have product
management and entrepreneurship principles.

Coming from a trading \& data science background myself, this will have
more of a quantitative \& strategic lean. It is my belief that concetps
like expected value, forecasting, and optimization are in fact very
relevant to product management. Design and aesthetics of are tremendous
importance and will be covered as well.

\bookmarksetup{startatroot}

\hypertarget{economic-principles}{%
\chapter{Economic Principles}\label{economic-principles}}

It is good to learn about the economic theory \& principles useful to an
entreprenuer. These are likelier to stand the test of time rather than
the latest product managment framework fad. Product Frameworks can't
significantly depeart from these deeper economic principles. Thinking of
oneself as an entereprenuer rather than just a ``product manager'' is a
great heurestic that will align you with your company and enable you to
make a plethora of good decisions.

\textbf{Companys Goal:}

A company is a group of people that build a product or service that adds
more value than the alternatives. (Including the status quo) This is the
`objective function' of capitalism in maximizing long term profit. Often
times these fundamentals can be forgottedn as PM's get lost in the day
to day work.

This outlook can have a beneficial effect in re-orientating your
thinking in helping \emph{others} improve. Note: To adopt this mindset
you do not need to completely drink the libertarian cool aid as
capitalism comes with tradeoffs. I.E. Your favorite restaurant you rave
about is a product of somebody elses labor and exemplifies the win/win
principle. However, if prior to construction, there was a small duck
pondthat was the beloved home to a flock of ducks, to build the
restaurant, the business will ruthlessly dispose of the pond to build
the restaurant. Capitalism comes with tradeoffs and this is where I
partially depart from the good profit book I cite.

Economic Decision Making Principles {[}1{]}

\begin{itemize}
\tightlist
\item
  Risk Appetite \& Expected Value

  \begin{itemize}
  \tightlist
  \item
    Project A has a 90\% chance of generating 100k while project B has a
    50\% chance of making 1M. While project B has a higher expected
    value, most employees will choose option A. (find citation)
  \end{itemize}
\item
  Opportunity Costs

  \begin{itemize}
  \tightlist
  \item
    Assuming the best choice is made, it is the ``cost'' incurred by not
    enjoying the benefit that would have been had by taking the second
    best available choice {[}2{]}
  \end{itemize}
\item
  Comparitive Advantage:

  \begin{itemize}
  \tightlist
  \item
    Comparative advantage is the ability to produce a good or service
    for a lower opportunity cost. {[}3{]} The power in comparitive
    advantage is understanding the difference with \emph{absolute
    advantage} as each person can potentially make a contribution, even
    if others can do everything better. As an example, envision a Doctor
    opens a new clinic. That doctor is a renaissance man that is also a
    great at taxation and billing. Should the Doctor both treat
    patienets and do the billing or should they focus on treating
    patients and outsource the billing work? {[}4{]}
  \item
    Comparative advantage can help inform how jobs can be constructed to
    play to the strenghts of the individual rather than fitting the
    individual into a pre-defined cookie cutter job.
  \end{itemize}
\item
  Sunk Costs

  \begin{itemize}
  \tightlist
  \item
    Unrecoverable past expenditure. To argue for decision many of us
    often say, ``well we already invested so much up until this point''.
  \end{itemize}
\end{itemize}

Sources:

\begin{itemize}
\tightlist
\item
  {[}1{]} Good Profit -
  https://www.amazon.com/Good-Profit-Charles-G-Koch-audiobook/ - pg70
\item
  {[}2{]} https://en.wikipedia.org/wiki/Opportunity\_cost
\item
  {[}3{]} https://www.thebalancemoney.com/comparative-advantage-3305915
\item
  {[}4{]} Good Profit -
  https://www.amazon.com/Good-Profit-Charles-G-Koch-audiobook/ - pg72
\end{itemize}

\bookmarksetup{startatroot}

\hypertarget{summary}{%
\chapter{Summary}\label{summary}}

In summary, this book has no content whatsoever.

\bookmarksetup{startatroot}

\hypertarget{section}{%
\chapter{}\label{section}}

\includegraphics{https://dan-olsen.com/wp-content/uploads/2022/01/PM-fit-pyramid-1024x588.png}

\bookmarksetup{startatroot}

\hypertarget{section-1}{%
\chapter{}\label{section-1}}

\bookmarksetup{startatroot}

\hypertarget{section-2}{%
\chapter{}\label{section-2}}

People Management Principles

\begin{itemize}
\tightlist
\item
  \textbf{Set clear expectations on roles, responsibilities and
  expectations}
\item
  \textbf{Before delegating, one must demonstrate competence in the task
  at hand}
\item
  \textbf{Comparative advantage and self selection}
\end{itemize}

\bookmarksetup{startatroot}

\hypertarget{section-3}{%
\chapter{}\label{section-3}}

\bookmarksetup{startatroot}

\hypertarget{design-principles}{%
\chapter{Design Principles}\label{design-principles}}

Before diving into design principles, a heurestic you can use as a PM
is, is this UI something you would be proud to attach your name too?

\bookmarksetup{startatroot}

\hypertarget{section-4}{%
\chapter{}\label{section-4}}

\bookmarksetup{startatroot}

\hypertarget{section-5}{%
\chapter{}\label{section-5}}

\bookmarksetup{startatroot}

\hypertarget{section-6}{%
\chapter{}\label{section-6}}

\bookmarksetup{startatroot}

\hypertarget{references}{%
\chapter*{References}\label{references}}
\addcontentsline{toc}{chapter}{References}

\markboth{References}{References}

\hypertarget{refs}{}
\begin{CSLReferences}{0}{0}
\end{CSLReferences}



\end{document}
